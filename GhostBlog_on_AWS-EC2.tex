%Ghost Blog on AWS EC2 free tier
%Sergio González

%\documentclass[a4paper]{article}
\documentclass[a4paper]{report}

\usepackage[english]{babel}
\usepackage{color}
\usepackage{fullpage}
\usepackage{url}

\title{Ghost Blog on AWS EC2 free tier}
\date{} %Excludes date
\author{Sergio Gonz\'alez}


\begin{document}

  \maketitle

  \begin{abstract}
    Abstract \ldots
  \end{abstract}

  \tableofcontents

  \section{Launch an EC2 instance}
    %\paragraph
    \begin{itemize}
      \item Select ``Free tier only'' and choose ``Amazon Linux AMI 2013.09.2'' 64 bit.
      \item Next: Configure instance details
      \item Next: Tag Instance.
        \begin{itemize}
          \item Name your instance, under the ``Value'' field.
        \end{itemize}
      \item Next: Configure security group
        \begin{itemize}
          \item Add Rule
          \begin{itemize}
            \item HTTP
            \item HTTPS
          \end{itemize}
        \end{itemize}
      \item Review and Launch
      \item Launch
        \begin{itemize}
          \item Create a new key pair
          \item Name it \& download it
        \end{itemize}
      \item Launch Instance
    \end{itemize}

  \section{Connect to the instance}
    \begin{itemize}
      \item Go to EC2 dashboard, Instances
      \item Select the instance and click ``Connect''
        \begin{itemize}
          \item chmod 400 yourKey.pem
          \item ssh -i yourKey.pem ec2-user@yourInstancePublicIP
        \end{itemize}
    \end{itemize}

  \section{Prepare the system on our instance}
    \subsection{Update the system}
      \begin{quotation}
        %sudo apt-get -y update \&\& sudo apt-get -y upgrade \&\& sudo apt-get -y dist-upgrade
        sudo yum update -y
      \end{quotation}

    \subsection{Install gcc compiler}
      \begin{quotation}
        %sudo apt-get -y install gcc %UBUNTU
        %sudo apt-get -y install build-essential %UBUNTU
        sudo yum install -y gcc

        sudo yum install -y gcc-c++ compat-gcc-32 compat-gcc-32-c++
      \end{quotation}

    \subsection{Install Node.js}
      \begin{quotation}
        wget http://nodejs.org/dist/node-latest.tar.gz

        tar -xzf node-latest.tar.gz

        rm node-latest.tar.gz

        cd node-directory

        ./configure

        make

        sudo make install
      \end{quotation}

    \subsection{Install Ghost}
      \begin{quotation}
        %cd

        sudo mkdir -p /var/www

        cd /var/www

        wget https://ghost.org/zip/ghost-latest.zip

        unzip -d ghost ghost-latest.zip

        rm ghost-latest.zip

        cd ghost

        sudo /usr/local/bin/npm install OR sudo /usr/local/bin/npm install --production
      \end{quotation}

    \subsection{Configure Ghost}
      \begin{quotation}
        vim config.example.js
        \begin{itemize}
          \item Under ``development'' and ``production:''
            \begin{itemize}
              \item Change ``host: '127.0.0.1','' to ``host: 'yourInstancePublicIP',''
              \item Change ``port: '2368','' to ``port: '80',''
            \end{itemize}
          \item
        \end{itemize}
        sudo /usr/local/bin/npm --production start
      \end{quotation}




\end{document}
